% Created 2022-05-14 Sat 10:32
% Intended LaTeX compiler: pdflatex
\documentclass[11pt]{article}
\usepackage[utf8]{inputenc}
\usepackage[T1]{fontenc}
\usepackage{graphicx}
\usepackage{grffile}
\usepackage{longtable}
\usepackage{wrapfig}
\usepackage{rotating}
\usepackage[normalem]{ulem}
\usepackage{amsmath}
\usepackage{textcomp}
\usepackage{amssymb}
\usepackage{capt-of}
\usepackage{hyperref}
\usepackage{amsthm}
\usepackage{mdframed}
\author{Jad Saklawi}
\date{\textit{[2018-07-23 Mon]}}
\title{Observations on Sound}
\hypersetup{
 pdfauthor={Jad Saklawi},
 pdftitle={Observations on Sound},
 pdfkeywords={},
 pdfsubject={},
 pdfcreator={Emacs 26.3 (Org mode 9.3.1)}, 
 pdflang={English}}
\begin{document}

\maketitle
\newtheorem*{observation}{Observation}
\newmdtheoremenv{observation}{}



\begin{observation}
Sound is a carrier of information, coordinates.
\end{observation}
\pagenumbering{gobble}

\begin{observation}
Sound drops sound.
\end{observation}

\begin{observation}
The auditory system in biological organisms is a sound antenna.
\end{observation}

\pagenumbering{gobble}

\newpage
\appendix

\section{Open Problems}
\label{sec:org43712c4}
\begin{itemize}
\item How much sound can a single sound wave carry.\footnote{Abstracts to: how much load can a single wave carry.}
\end{itemize}


\pagenumbering{gobble}
\end{document}
