% Created 2022-05-16 Mon 18:02
% Intended LaTeX compiler: pdflatex
\documentclass[11pt]{article}
\usepackage[utf8]{inputenc}
\usepackage[T1]{fontenc}
\usepackage{graphicx}
\usepackage{grffile}
\usepackage{longtable}
\usepackage{wrapfig}
\usepackage{rotating}
\usepackage[normalem]{ulem}
\usepackage{amsmath}
\usepackage{textcomp}
\usepackage{amssymb}
\usepackage{capt-of}
\usepackage{hyperref}
\usepackage{amsthm}
\usepackage{mdframed}
\author{Jad Saklawi}
\date{\textit{[2022-05-14 Sat]}}
\title{Of Waves}
\hypersetup{
 pdfauthor={Jad Saklawi},
 pdftitle={Of Waves},
 pdfkeywords={},
 pdfsubject={},
 pdfcreator={Emacs 26.3 (Org mode 9.3.1)}, 
 pdflang={English}}
\begin{document}

\maketitle
\newtheorem*{observation}{Observation}
\newmdtheoremenv{observation}{}
\theoremstyle{definition}
\newtheorem{definition}{Definition}[section]

\theoremstyle{example}
\newtheorem{example}{Example}[section]

\section{Observations}
\label{sec:orgb0eba0c}

\begin{observation}
\subsection{Waves are carriers of information, coordinates.}
\label{sec:org989a148}
\end{observation}

\begin{observation}
\subsection{Given a wave burst emitted on an antenna, exactly one wave is read.}
\label{sec:orgf365633}
\end{observation}

\begin{observation}
\subsection{Given a wave burst emitted on a multi-polar antenna, wave is read  by pole closest to burst. \footnote{Planet Earth, wave always read by the right antenna for the case of equally distant antennas.}}
\label{sec:orgfc142ce}
\end{observation} 

\begin{observation}
\subsection{Waves are carriers of information, an identifier.}
\label{sec:org0601300}
\subsubsection{Waves emitted by the same burst carry the same identifier.}
\label{sec:org05d489d}
\end{observation}

\section{Definitions}
\label{sec:orgd5524ae}
\begin{definition}{Wave Burst:}
The collection of waves emitted by the same trigger.
\end{definition}
\begin{example}
Consider a click sound. Wave burst would be collection of waves emitted carrying the sound "click".
\end{example}
\section{Open Problems}
\label{sec:orgb0de525}
\begin{itemize}
\item Number of waves emitted per burst.
\item Direction of waves emitted by a burst.\footnote{For example emitted in a beam, spherical, random, etc..}
\end{itemize}
\end{document}
